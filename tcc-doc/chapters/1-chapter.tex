% ----------------------------------------------------------
\chapter{Introdução}
% ----------------------------------------------------------

A discussão acerca do gerenciamento de resíduos sólidos adquiriu potência em meados de 1970, acompanhando os tópicos das conferências como na de Estocolmo (1972), Tbilisi (1977) e ECO 92 (1992). De 1993 a 2013 a produção científica no mundo relacionada ao tópico triplicou e seguiu duplicando nos anos de 2003 a 2013 \cite{deus_residuos_2015}. Após 30 anos, podemos ver nas bases de dados de produções científicas que essa tendência continua.

Ainda que amplo seja o estudo e debate acerca das questões ambientais, observa-se que na visão popular os resíduos ainda são associados a uma imagem negativa — restos, sujeira, incômodo —, o que pode dificultar a criação de estratégias pelo governo para uma destinação sustentável desses resíduos \cite{santiago_gestao_2016}.

No Brasil, seguindo as legislações europeias 1999/31/CE \cite{noauthor_directiva_1999} e 2008/98/EC \cite{noauthor_directive_2018}, foi publicada a \gls{PNRS} que conforme descrito no Art 1º:

\begin{citacao}
	Esta lei institui a Política Nacional de Resíduos Sólidos, dispondo sobre seus princípios, objetivos e instrumentos, bem como sobre as diretrizes relativas à gestão integrada e ao gerenciamento de resíduos sólidos, incluídos os perigosos, às responsabilidades dos geradores e do poder público e aos instrumentos econômicos aplicáveis. \cite[Art. 1º]{brasil_lei_nodate}.
\end{citacao}

Em 2022, o DECRETO Nº 10.936 \cite{brasil_decreto_2022} avançou nas definições de responsabilidades compartilhadas dos envolvidos no ciclo de vida do produto — fabricantes, importadores, distribuidores, comerciantes, consumidores e os titulares dos serviços públicos de limpeza urbana e de manejo de resíduos sólidos — mencionando marjoritariamente \gls{RSU}.

No que tange aos \gls{RSI}, observa-se que apesar da existência de bases de dados descentralizadas pertinentes à geração dos resíduos, ainda carece de um fluxo claro, objetivo e unificado para lidar com a problemática.

Com isso, entende-se a importância do desenvolvimento de alternativas para a questão do direcionamento dos \gls{RSI}s, e na tentativa de preencher uma lacuna na cadeia produtiva baseada no descarte inconsciente e irresponsável, este trabalho propõe uma aplicação que reuna dados disponíveis sobre \gls{RSI}s a fim de conectar geradores de resíduos e potenciais consumidores de resíduos no âmbito industrial. Isso segue as diretrizes do \gls{PNRS} sobre logística reversa e economia circular. 

Como \gls{SC} tem sido destaque na destinação de resíduos sólidos \cite{crea_sc_destino_2013}, considerou-se válido o foco do trabalho para o estado. Contou-se com a ajuda do \gls{IMA/SC} para obtenção dos dados de \gls{MTR} em \gls{SC} para o desenvolvimento do projeto.

% ----------------------------------------------------------
\section{Objetivos}
% ----------------------------------------------------------

Nas seções abaixo estão descritos o objetivo geral e os objetivos específicos deste TCC.

% ----------------------------------------------------------
\subsection{Objetivo Geral}
% ----------------------------------------------------------

A proposta deste trabalho está vinculada aos \gls{ODS} da \gls{ONU}, em particular com o \gls{ODS} 9 — Indústria Inovação e Infraestrutura. Entrando no escopo deste projeto os ítens 9.4, 9.5 e 9.c \cite{noauthor_sustainable_nodate}, que dizem a respeito a:

\begin{citacao}
	9.4 Até 2030, modernizar a infraestrutura e reabilitar as indústrias para torná-las sustentáveis, com eficiência aumentada no uso de recursos e maior adoção de tecnologias e processos industriais limpos e ambientalmente corretos; com todos os países atuando de acordo com suas respectivas capacidades.
\end{citacao}

\begin{citacao}
	9.5 Fortalecer a pesquisa científica, melhorar as capacidades tecnológicas de setores industriais em todos os países, particularmente os países em desenvolvimento, inclusive, até 2030, incentivando a inovação e aumentando substancialmente o número de trabalhadores de pesquisa e desenvolvimento por milhão de pessoas e os gastos público e privado em pesquisa e desenvolvimento
\end{citacao}

\begin{citacao}
	9.c Aumentar significativamente o acesso às tecnologias de informação e comunicação e se empenhar para oferecer acesso universal e a preços acessíveis à internet nos países menos desenvolvidos, até 2020
\end{citacao}

% ----------------------------------------------------------
\subsection{Objetivos Específicos}
% ----------------------------------------------------------

Particularmente, neste trabalho pretende-se alcançar os seguintes objetivos:
\begin{enumerate}
    \item Levantar a viabilidade e/ou potencial de um produto de software destinado ao redirecionamento de \gls{RSI}s em \gls{SC};
	\item Coletar, analisar e tratar os dados de geração de resíduos sólidos de relatórios de \gls{MTR} providos pelo \gls{IMA/SC};
	\item Propor conceitos de um sistema que conecte potenciais consumidores de \gls{RSI} aos respectivos geradores em \gls{SC};
	\item Desenvolver um protótipo de aplicativo web com mínimas funcionalidades utilizando tecnologias de código aberto;
	\item Promover uma reflexão sobre o direcionamento de resíduos sólidos no estado e a reinserção dos mesmos na cadeia produtiva.
\end{enumerate}