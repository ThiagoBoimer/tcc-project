% ----------------------------------------------------------
\chapter{Conclusão}
% ----------------------------------------------------------

No decorrer deste trabalho, entendeu-se que nos últimos 10 anos o poder público investiu em instrumentos que coletam informações sobre resíduos industriais, como o \gls{RAPP}, o Inventário de Resíduos Sólidos Industriais, as Declarações Anuais de Resíduos Sólidos, o programa Lixão Zero, o desenvolvimento do \gls{SINIR} com intuito de agregar todos esses dados, e então o sistema \gls{MTR}. 

Diante disso, a concepção de um produto voltado ao aproveitamento de resíduos sólidos possui inúmeros desafios, pois para alcançar às necessidades de um mercado cada vez mais acelerado as informações devem estar claras, breves e concisas. Nesse sentido, a adoção do \gls{MTR} como base para o sistema aqui discutido prevê que maiores esforços sejam colocados pelos governos estaduais para a disseminação do preenchimento responsável dos manifestos, a acessibilidade dos dados para todos e o estímulo ao empreendedorismo dentro deste mercado.

A proposta do aplicativo web, nomeado “Residuose”, abrangiu dois conceitos, o primeiro para se ter um protótipo funcional disponível para conhecer as dificuldades relacionadas a construção de um sistema desse tipo. Já o segundo foi o resultado do amadurecimento de ideias e expectativas anteriores ao desenvolvimento, considerando-se como uma próxima etapa, mas que ainda depende de maior disponibilidade de acesso aos dados de \gls{MTR}, recursos financeiros, pessoas capacitadas e tempo para o trabalho. De qualquer forma, viu-se através da análise dos dados a existência de uma preocupação notável em \gls{SC} com a destinação dos resíduos, o que é um indicativo de quê um sistema facilitador e incremental às indústrias seja bem-vindo. Para futuros trabalhos, sugere-se:
\begin{itemize} 
	\item A elaboração de pesquisas direcionadas à caracterização de resíduos sólidos de acordo com as classificações do \gls{IBAMA}, para que se tenham maiores informações sobre o material, tornando-o mais visível para empresas interessadas;
	\item O estudo da utilização de \gls{LLM}s no âmbito da Ciência de Materiais para o levantamento, através de artigos acadêmicos, dos potenciais resíduos substitutos de matérias-primas na indústria e como eles se relacionam com as classificações existentes de \gls{RSI}s; 
	\item A busca por fundos de fomento ao empreendedorismo acadêmico para que o projeto tenha recursos, orientação empresarial, levando à capacidade de crescimento das pessoas envolvidas e do produto;
	\item Um melhor alinhamento com o \gls{IMA/SC} para obtenção de dados numa menor granularidade e com maior riqueza de informações;
	\item Um estudo de caso de uma combinação entre duas empresas para que o aplicativo lide com uma situação real revelando problemáticas não abordadas neste documento.
\end{itemize}